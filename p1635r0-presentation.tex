\documentclass[dvipsnames,10pt]{jbeamer}

% Insititution Name
\title[P1635R0]{P1635R0 : A Design for an Inter-Operable and Customizable Linear
Algebra Library}
\author{Jayesh Badwaik}
\date{SG14 Linear Algebra SIG}
\begin{document}
\maketitle

\begin{frame}
  \frametitle{What this paper is trying to be?}

  \begin{itemize}
    \item Not a library proposal.
    \item Rather a collection of concepts/techniques which enable the library
      writers to write inter-operable libraries.
  \end{itemize}
\end{frame}

\begin{frame}
  \frametitle{Linear Algebra}
  \framesubtitle{Why think about Inter-Operability between Linear Algebra
  Libraries?}

  \vspace{0.5cm}
  \begin{itemize}
    \item Mixing of \inlinecode{vector} in current code.
    \item Standard will always move slowly with regards to current practice.
    \item There will be other linear algebra libraries which will do things
      standard cannot do.
    \item Some libraries will use these \textbf{other} libraries.
    \item These other libraries \textbf{might not} use the vocabulary-like
      types.
    \item Well-designed linear algebra types in STL will prevent drive-by
      implementations of library, but custom library implementers might still go
      off and do their own thing.
  \end{itemize}
\end{frame}

\begin{frame}
  \frametitle{Impacts of Non-Interoperability on User Code}
      \begin{itemize}
        \item Run-Time Performance

            If I use a library which uses standard linear algebra library and
            another library which uses another standard library, then, there
            will always be a copy between the two.

            This will result in:
            \begin{itemize}
              \item Reimplementing the library by performance-conscious people.
              \item Lack of performance by copies permeating through the code
                base.
            \end{itemize}
    \item  Ecosystem Effects
      \begin{itemize}
        \item There is a possibility (possibly small) of the standard linear
          algebra ecosystem falling enough behind that people stop using it, and
          use other libraries instead.
      \end{itemize}
    \end{itemize}

\end{frame}

\begin{frame}[fragile]
  \frametitle{Defining the Problem Space}
  \framesubtitle{Restriction to Operator Algebra}

  \vspace{1cm}

\begin{acodecpp}{}
auto const a = alpha::func(); // std::matrix
auto const b = beta::func(); // b:matrix:
auto const c = beta::other_func(); // b::vector:
auto const d = p::compute_eigenvalues(a,b);
auto const e = p::gaussian_elimination(a,c);
\end{acodecpp}


What do we not have today that prevents this from happening?

\begin{itemize}
  \item \inlinecode{a+b}, \inlinecode{a*b}, \inlinecode{a[i] * b[j]} (vectors)
  \item Pessimization of operators in face of structured matrices, sparse matrices etc.
\end{itemize}

\end{frame}

\begin{frame}[fragile]
  \frametitle{Defining the Problem Space : Compatibility With?}
  \framesubtitle{Heterogenous Computing}

Talk a little heterogenous computing.
\begin{acodecpp}{}
auto const a = alpha::func(); // std::matrix
auto const c = beta::other_func(); // b::vector:
// Compiler Error?
auto const e = gpu::gaussian_elimination(a,c);
// Copy and Compute (What if a and c are already on gpu?)
auto const f = gpu::gaussian_elimination(a,c);
\end{acodecpp}
\end{frame}

\begin{frame}[fragile]
  \frametitle{Defining the Problem Space : Compatibility With?}
  \framesubtitle{Expression Templates}

\begin{acodecpp}{}
auto const a = alpha::func(); // std::matrix
auto const t1 = beta::transpose(a); // ET?
auto const t2 = beta::transpose(t1*alpha::g()); // Return By Value
\end{acodecpp}
\end{frame}

\begin{frame}[fragile]
  \frametitle{Tools to Solve the Problem}
  \framesubtitle{Three Techniques}

  \begin{itemize}
    \item  Non-owning Types aka Expression Templates / Views
    \item  Engine-Aware Types
    \item  Engines and Engine-Associated Operator
  \end{itemize}
\end{frame}

\begin{frame}[fragile]
  \frametitle{1. Non-owning Types aka ET/Views}
  \framesubtitle{Intent : Not Wording}

  A type is said to be a non-owning type if an object of this type needs an
  independent object to exist separately for it to be well-behaved.

\begin{acodecpp}{}
auto a = alpha::func(); // Returned by Value
auto view= alpha::transpose(a); // A Non-owning Type
\end{acodecpp}

So, all such types should export

\inlinecode{using is\_owning\_type=std::false\_type;}
\end{frame}

\begin{frame}[fragile]
  \frametitle{1. Non-owning Types aka ET/Views}
  \framesubtitle{Intent : Not Wording}

Assume \inlinecode{alpha::func()} returns \inlinecode{a::matrix}.
Now, there exists a proof of concept which does the following
\begin{acodecpp}{}
// returns by value
auto const a = alpha::tp(alpha::func());
// returns a view
auto const b = alpha::tp(a);
// returns a view
auto const c = alpha::tp(alpha::tp(b));
\end{acodecpp}
\end{frame}


\begin{frame}[fragile]
  \frametitle{2. Engine-Aware Types}
  \framesubtitle{Engine-Aware Non-Owning Types}


\begin{acodecpp}{}
 template<typename E, typename OT>
  class view {
  public:
    using engine_type = E;
    using is_owning_type = std::false_type;

  public:
    template <typename NE>
    auto change_engine() const -> view<NE, OT>;

  private:
    OT* ot_;
  };
\end{acodecpp}
\end{frame}

\begin{frame}[fragile]
  \frametitle{2. Engine-Aware Types}
  \framesubtitle{Engine-Aware Owning Types}


\begin{acodecpp}{}
  template<typename E>
  class owning {
  public:
    using engine_type = E;
    using is_owning_type = std::true_type;

  public:
    template <typename NE>
    auto change_engine() && -> owning<NE>;

    template <typename NE>
    auto change_engine() & -> view<NE, owning<E>>;

    template <typename NE>
    auto change_engine() const& -> view<NE, owning<E> const>;
  };
\end{acodecpp}
\end{frame}

\begin{frame}[fragile, allowframebreaks]
  \frametitle{3. Engine and Engine-Associated Operators}
  \framesubtitle{\inlinecode{std::math} namespace}

\begin{acodecpp}{An Engine}
  struct serial_cpu_engine{};
\end{acodecpp}


\begin{codecpp}{Addition Engine Traits}
template <typename T, typename U>
struct addition_traits {
  using result_type = X ; // X can be determined using any logic required
};
\end{codecpp}

\begin{codecpp}{Engine-Based Operator}
template <
typename T,
typename U,
typename = std::enable_if<
std::math::uses_engine_v<serial_cpu_engine, T, U>>>
auto operator+(T&& t, U&& u) -> addition_traits_r<T&&, U&&>;
\end{codecpp}

\begin{codecpp}{Serial Code for Multiplication of Two Vectors}
  using sce = std::math::serial_cpu_engine;
  using stdvec = std::math::vector;
  using custvec = another_lib::vector;

  auto const ov1 = stdvec<sce>(arg...);
  auto const ov2 = custvec<sce>(arg...);

  auto const view_3 = ov1 + ov2;

  auto const ov4 = stdvec<sce>(args...) + ov1;
\end{codecpp}

\begin{itemize}
\item Discuss the ADL-based lookup and how we can ensure that it does not go
    rogue.
\item DO NOT DECLARE OPERATORS FOR TYPES, ONLY FOR ENGINES.
\end{itemize}


\begin{codecpp}{Parallel Code for Multiplication of Two Vectors}
  using sce = std::math::sce;
  using pce = custom::pce;
  using stdvec = std::math::vector;
  using custvec = another_lib::vector;

  auto const ov1 = stdvec<sce>(arg...);
  auto const ov2 = custvec<pce>(arg...);

  auto const view_3 = ov1.change_engine<pce>() + ov2;

  auto const ov4 = f().change_engine<pce>()
                    + ov2.change_engine<pce>();
\end{codecpp}




\begin{codecpp}{Inheriting Behavior of Operator from Another Engine}
template <
typename T,
typename U,
typename = std::enable_if<
  std::math::uses_engine_v<serial_cpu_engine, T, U>>>
auto operator+(T&& t, U&& u) -> addition_traits_r<T&&, U&&>
{
  return t.change_engine<some_other>()
          + u.change_engine<some_other>();
}
\end{codecpp}
\end{frame}

\begin{frame}
  \frametitle{Some Additional Points}

  \begin{itemize}
    \item Less pressure on standard library to have all functions.
    \item Graceful migration of functions from one library to another.
    \item A weak standard library will spawn a creation of lot of "engines" and
      types. But with this design, a growth of  stronger standard library can
      potentially reduce the problem.
    \item Question of well-established practice: None of the Linear Algebra
      library has used this practice before, but STL is based on this idea.
      So, C++ wise, the methods are pretty established.
    \item Need to check compatibility of these techniques with linear algebra.
  \end{itemize}
\end{frame}



\begin{frame}
  \frametitle{So, we can (potentially) solve our problems this way.}

  \begin{itemize}
    \item Proof of concept at
      \url{https://github.com/liblac/proof-of-concept}

    \item Questions/Polls: Does SG14 consider this a viable exploration of
      design for linear algebra library?
      (Implying that current design be modified in accordance with these ideas.)
  \end{itemize}
\end{frame}




\end{document}
